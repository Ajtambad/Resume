%%%%%%%%%%%%%%%%%%%%%%%%%%%%%%%%%%%%%%%%%%%%%%%%%%%%%%%%
% Author: Vignesh Iyer                                 %
% MS CSE ASU                                           %
%%%%%%%%%%%%%%%%%%%%%%%%%%%%%%%%%%%%%%%%%%%%%%%%%%%%%%%%

% -------------- README --------------
% Visit https://github.com/vgnshiyer/ASU-sparkysundevil-resume-template for comprehensive documenation.

% -------------- Resume ---------------
\documentclass{resume}

\begin{document}

% --------- Contact Information -----------
\introduction[
    fullname={Amogh Jagadish Tambad},
    email={tambadamogh@gmail.com},
    phone={(480) 876-5096},
    linkedin={linkedin.com/in/ajtambad},
    github={github.com/Ajtambad}
]

% --------- Education ---------
\begin{educationSection}{Education}
    \educationItem[
        university={Arizona State University, Tempe, AZ},
        graduation={May 2025},
        grade={4.00 GPA},
        program={Master of Science, Computer Science},
        coursework={Cloud Computing, Data Processing at Scale, Data Mining, Software Security},
    ]

    \educationItem[
        university={REVA University, Bangalore, India},
        graduation={May 2021},
        grade={3.77 GPA},
        program={Bachelor of Technology, Computer Science},
        coursework={Data Structure and Algorithms, Operating Systems, Cloud Computing},
    ]
\end{educationSection}
% --------- Skills -----------
\begin{skillsSection}{Skills}
    \skillItem[
        category={Languages},
        skills={Python, C++, Bash, SQL, YAML, Scala, HTML, JavaScript, Java}
    ] \\
    \skillItem[
        category={Technologies},
        skills={AWS, GCP, Linux, PowerShell, PostmanAPI, Splunk, Cribl, Zabbix, Chef, Ansible, Terraform, Docker, Kubernetes, Jenkins, Github Actions, Nginx, Gunicorn}
    ] \\
    \skillItem[
        category={Frameworks},
        skills={PyTorch, TensorFlow, scikit-learn, Flask, FastAPI, React, Node, Next.js}
    ] \\
    \skillItem[
        category={Data},
        skills={PostgreSQL, MongoDB, Kafka, Spark, Hadoop, ZooKeeper, Prometheus, Grafana, JSON}
    ]
\end{skillsSection}

% --------- Experience -----------
\begin{experienceSection}{Professional Experience}
    \experienceItem[
        company={Arch Mortgage Insurance},
        location={Greensboro, NC},
        position={IT-Infrastructure-Platform/Site Reliability Engineer Intern},
        duration={Jun 2024 - Aug 2024}
    ]
    \begin{itemize}
        \itemsep -6pt {}
        \item Filtered and routed logs from OpenShift Kubernetes clusters to Splunk using Cribl Stream pipelines, reducing \textbf{daily Splunk storage usage by 40–50 GB} and improving \textbf{log search performance by 20\%}.
        \item Enhanced log clarity and parsing efficiency by leveraging Cribl Parser and Mask functions, resulting in a streamlined raw field and reducing parsing time to 2–3 seconds per log.
        \item Designed an automated pipeline using Ansible, Skopeo, and Red Hat registry APIs to sync updated catalog images to Nexus Repository, reducing manual update time by\textbf{ 90\%}.
        
    \end{itemize}

    \experienceItem[
        company={Cerner Healthcare},
        location={Bangalore, India},
        position={System Engineer - 1},
        duration={May 2021 - Jul 2023}
    ]
    \begin{itemize}
        \itemsep -6pt {}
        \item Migrated 80\% of data from on-prem to AWS, enhancing data access flexibility, security, and cost-efficiency.
        \item Participated in regular on-call rotations, leveraging Zabbix and Splunk for system health monitoring, troubleshooting server issues, and resolving production alerts within 15 minutes, maintaining 99.99\% service reliability.
        \item Troubleshot and resolved Jenkins pipeline issues, minimizing support ticket resolution time by 40\% and ensuring 99.9\% uptime for CI/CD workflows, leading to uninterrupted deployment pipelines.
        \item Managed 300+ bi-weekly microservice deployments, including Splunk and non-Splunk-based services, using Chef, accelerating delivery of new UI and backend features in a fast-paced production environment.
    \end{itemize}

\end{experienceSection}

% --------- Projects -----------
\begin{experienceSection}{Academic projects}
    \projectItem[
        title={End-to-End Deployment Automation},
        duration={Mar 2025 - Apr 2025},
        % keyHighlight=Collaborated in a team of three to design model of custom hand cycle for polio victims (SOLIDWORKS).
    ]
    \begin{itemize}
        \vspace{-0.5em}
        \itemsep -6pt {}
        \item Automated end-to-end AWS EC2 provisioning using Terraform, Ansible, Jenkins, and GitHub Actions, enabling reproducible infrastructure setup and hands-free web service deployments.
        \item Built and optimized CI/CD pipelines to dynamically retrieve instance IPs, configure secure SSH access, and deploy services, eliminating manual intervention and resolving IAM and resource issues in production-like environments.
    \end{itemize}

    \projectItem[
        title=Kubernetes based Data Processing Pipeline,
        duration={Oct 2024 - Nov 2024},
    ]
    \begin{itemize}
        \vspace{-0.5em}
        \itemsep -6pt {}
        \item Designed and deployed a scalable, near-real-time data pipeline on Kubernetes for spatial analytics of NYC Taxi Rides, enabling data-driven decisions for urban mobility by processing ride patterns efficiently.
        \item Leveraged Kafka, Kafka Connect, ZooKeeper, and Neo4j for real-time ingestion and graph processing (PageRank, BFS), uncovering location importance and optimizing resource allocation.
    \end{itemize}

    \projectItem[
        title=AWS-Based Face Recognition App,
        duration={Feb 2024 - May 2024},
    ]
    \begin{itemize}
        \vspace{-0.5em}
        \itemsep -6pt {}
        \item Developed and deployed a Flask-based image recognition app using Gunicorn on AWS EC2, enabling HTTP-based uploads and forwarding images to S3 via SQS for asynchronous processing.
        \item Designed an auto-scaling app tier that scaled up to 20 EC2 instances based on SQS queue depth, ensuring efficient, real-time image processing under varying workloads.
    \end{itemize}

    \projectItem[
        title=RAG Implementation for arXiv Papers,
        duration={Oct 2024 - Nov 2024},
    ]
    \begin{itemize}
        \vspace{-0.5em}
        \itemsep -6pt {}
        \item Extracted and vectorized multimodal content (text, tables, images, equations) from 2000+ arXiv papers using CLIP and text embedding models, storing results in separate vector stores and indexing with DynamoDB.
        \item Built a semantic retrieval pipeline with top‑k similarity search and GPT‑4o mini-based summarization, delivering concise, context-aware answers to user queries.
    \end{itemize}
    
\end{experienceSection}


\end{document}
