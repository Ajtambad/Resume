%%%%%%%%%%%%%%%%%%%%%%%%%%%%%%%%%%%%%%%%%%%%%%%%%%%%%%%%
% Author: Vignesh Iyer                                 %
% MS CSE ASU                                           %
%%%%%%%%%%%%%%%%%%%%%%%%%%%%%%%%%%%%%%%%%%%%%%%%%%%%%%%%

% -------------- README --------------
% Visit https://github.com/vgnshiyer/ASU-sparkysundevil-resume-template for comprehensive documenation.

% -------------- Resume ---------------
\documentclass{resume}
\usepackage{amsmath}
\begin{document}

% --------- Contact Information -----------
\introduction[
    fullname={Amogh Jagadish Tambad},
    location = {San Francisco, CA},
    email={tambadamogh@gmail.com},
    phone={(480) 876-5096},
    linkedin={linkedin.com/in/ajtambad},
    github={github.com/Ajtambad}
]

% --------- Education ---------
\begin{educationSection}{Education}
    \educationItem[
        university={Arizona State University, Tempe, AZ},
        graduation={May 2025},
        grade={4.00 GPA},
        program={Master of Science, Computer Science},
        coursework={Cloud Computing, Data Processing at Scale, Data Mining, Software Security},
    ]

    \educationItem[
        university={REVA University, Bangalore, India},
        graduation={May 2021},
        grade={3.77 GPA},
        program={Bachelor of Technology, Computer Science},
        coursework={Data Structure and Algorithms, Operating Systems},
    ]
\end{educationSection}
% --------- Skills -----------
\begin{skillsSection}{Skills}
    \skillItem[
        category={Programming Languages},
        skills={Python, C++ Go, Java, JavaScript, Typescript, Scala, HTML, XML}
    ] \\
    \skillItem[
        category={Technologies},
        skills={AWS (EC2, Lambda, S3, ECS, CloudFormation, EKS, ECR), TensorFlow, Docker, Kubernetes, Terraform, Knative, Jenkins, Github Actions, Chef React, Angular, Flask, Node.js}
    ] \\
    \skillItem[
        category={Tools},
        skills={Linux (RHEL, Ubuntu), Git (Version Control), Github, Ansible, Gitlab, Nginx, Kafka, Redis, Prometheus, Postman, Cursor, CoPilot, Claude Code}
    ] \\
    \skillItem[
        category={Database Systems},
        skills={SQL(MySQL, PostgreSQL), NoSQL (MongoDB)}
    ] \\
    \skillItem[
        category={LLM/GenAI Frameworks},
        skills={MCP, LangChain, LangGraph, RAG, VectorDBs (FAISS, ChromaDB)}
    ] \\
    \skillItem[
        category={Miscellaneous},
        skills={Distributed Systems, RESTful APIs, Microservices Architecture, Object-Oriented Programming, Agile, Test Driven Development (TDD), Infrastructure-as-code, Cloud Infrastructure, Communication Skills, Collaboration, Writing}
    ]
\end{skillsSection}

% --------- Experience -----------
 \begin{experienceSection}{Work Experience}
    \experienceItem[
        company={Arizona State University},
        location={Remote},
        position={Research Assistant, VISA Lab},
        duration={Jun 2025 - Present}
    ]
    \begin{itemize}
        \itemsep -6pt {}
        \item Developed \textbf{FlowBench}, a workflow-based distributed benchmark by leveraging \textbf{Python}, \textbf{Docker}, \textbf{Kubernetes}, and \textbf{FaaS} principles to evaluate custom edge computing applications, providing a comprehensive report on 6+ metrics.
        \item Built and tested a video analytics workflow via \textbf{OpenCV} on a containerized microservices architecture with Kubernetes, processing 10,000+ frames per minute.
    \end{itemize}

    \experienceItem[
        company={Arch Mortgage Insurance},
        location={Greensboro, NC},
        position={Site Reliability Engineer (SRE) Intern},
        duration={Jun 2024 - Aug 2024}
    ]
    \begin{itemize}
        \itemsep -6pt {}
        \item Filtered and routed logs from OpenShift Kubernetes clusters to Splunk using Cribl Stream pipelines, reducing daily Splunk data ingestion by \textbf{40–50 GB} and improving \textbf{log search performance by 20\%}
        \item Designed an automated pipeline using Ansible and Red Hat registry APIs to sync updated catalog images to Nexus Repository, reducing manual update time by \textbf{ 90\%}
        
    \end{itemize}

    \experienceItem[
        company={Cerner Healthcare (Oracle Health)},
        location={Bangalore, India},
        position={System Engineer - 1},
        duration={May 2021 - Jul 2023}
    ]
    \begin{itemize}
        \itemsep -6pt {}
        \item Migrated 80\% of data from on-prem to \textbf{AWS}, enhancing data access flexibility, security, and cost-efficiency
        \item Participated in regular on-call rotations with cross-functional teams, leveraging \textbf{Zabbix} and \textbf{Splunk} for system health monitoring, quality assurance, and resolving production alerts, maintaining \textbf{99.99\%} service reliability
        \item Troubleshot and performed root cause analysis \textbf{Jenkins} pipeline issues, minimizing support ticket resolution time by \textbf{40\%} and ensuring \textbf{99.9\%} uptime for \textbf{CI/CD} workflows, leading to uninterrupted deployment pipelines
        \item Managed \textbf{300+} bi-weekly microservice deployments, including Splunk and non-Splunk-based services, using \textbf{Chef}, while also tracking them with \textbf{JIRA}, accelerating delivery of new UI and backend features in a fast-paced production environment
    \end{itemize}

\end{experienceSection}

% --------- Projects -----------
\begin{experienceSection}{Academic projects}

    \projectItem[
        title={End-to-End Deployment Automation},
        duration={Apr 2025},
    ]
    \begin{itemize}
        \vspace{-0.5em}
        \itemsep -6pt {}
        \item Automated end-to-end AWS EC2 provisioning and cloud infrastructure management using \textbf{Terraform}, \textbf{Ansible}, \textbf{Jenkins}, and \textbf{GitHub Actions}, enabling infrastructure setup and hands-free web service deployments based on version control changes.
        \item Built and optimized CI/CD pipelines to dynamically retrieve instance IPs, configure secure SSH access, and deploy services, eliminating manual intervention and resolving IAM and resource issues in production-like environments
    \end{itemize}

    \projectItem[
        title=AWS-Based Face Recognition App,
        duration={May 2024},
    ]
    \begin{itemize}
        \vspace{-0.5em}
        \itemsep -6pt {}
        \item Developed and deployed a \textbf{Flask}-based image recognition app using \textbf{Gunicorn} on AWS EC2, enabling HTTP REST API based uploads and forwarding images to \textbf{S3} via \textbf{SQS} for asynchronous processing using AWS service API.
        \item Designed an auto-scaling app tier that scaled up to 20 EC2 instances based on \textbf{SQS} queue depth, ensuring efficient, real-time image processing under varying workloads
    \end{itemize}
    
\end{experienceSection}


\end{document}
