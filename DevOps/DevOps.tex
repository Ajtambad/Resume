%%%%%%%%%%%%%%%%%%%%%%%%%%%%%%%%%%%%%%%%%%%%%%%%%%%%%%%%
% Author: Vignesh Iyer                                 %
% MS CSE ASU                                           %
%%%%%%%%%%%%%%%%%%%%%%%%%%%%%%%%%%%%%%%%%%%%%%%%%%%%%%%%

% -------------- README --------------
% Visit https://github.com/vgnshiyer/ASU-sparkysundevil-resume-template for comprehensive documenation.

% -------------- Resume ---------------
\documentclass{resume}
\usepackage{amsmath}
\begin{document}

% --------- Contact Information -----------
\introduction[
    fullname={Amogh Jagadish Tambad},
    location = {Tempe, AZ},
    email={tambadamogh@gmail.com},
    phone={(480) 876-5096},
    linkedin={linkedin.com/in/ajtambad},
    github={github.com/Ajtambad}
]

% --------- Education ---------
\begin{educationSection}{Education}
    \educationItem[
        university={Arizona State University, Tempe, AZ},
        graduation={May 2025},
        grade={4.00 GPA},
        program={Master of Science, Computer Science},
    ] % ASU

    \educationItem[
        university={REVA University, Bangalore, India},
        graduation={May 2021},
        grade={3.77 GPA},
        program={Bachelor of Technology, Computer Science},
    ]
\end{educationSection}

\begin{certificateSection}{Certifiations}
    \begin{itemize}
        \item AWS Certified Solutions Architect – Associate (SAA-C03)
    \end{itemize} 
\end{certificateSection}

% --------- Experience -----------
 \begin{experienceSection}{Professional Experience}
    \experienceItem[
        company={DriverAI LLC},
        location={Remote},
        position={Software Engineer},
        duration={Dec 2025 - Present}
    ]
    \begin{itemize}
        \itemsep -6pt {}
        \item Engineered \textbf{Terraform} automation pipelines for \textbf{AWS} infrastructure provisioning, implementing security controls and best practices across multiple services
        \item Implemented gamification features with \textbf{FastAPI} backend and \textbf{MySQL} database to increase user engagement and data collection through reward-based incentives
    \end{itemize}


    \experienceItem[
        company={Arch Mortgage Insurance},
        location={Greensboro, NC},
        position={Site Reliability Engineer (SRE) Intern},
        duration={Jun 2024 - Aug 2024}
    ]
    \begin{itemize}
        \itemsep -6pt {}
        \item Architected log filtering and routing pipelines using \textbf{Cribl Stream} for OpenShift \textbf{Kubernetes} clusters, reducing Splunk ingestion by 40-50 GB/day (\textbf{~\$32K annual savings} ) and improving search performance by \textbf{20\%}
        \item Automated container image synchronization between Red Hat Registry and Nexus Repository using \textbf{Ansible} and REST APIs, eliminating \textbf{90\%} of manual update overhead and ensuring consistent artifact versions 
  across environments
        
    \end{itemize}

    \experienceItem[
        company={Cerner Healthcare (Oracle Health)},
        location={Bangalore, India},
        position={System Engineer},
        duration={May 2021 - Jul 2023}
    ]
    \begin{itemize}
        \itemsep -6pt {}
        \item Led migration of \textbf{500TB+} enterprise healthcare data from on-prem infrastructure to AWS using \textbf{Python} and \textbf{Bash} scripts and AWS \textbf{S3}, reducing storage costs by \textbf{30\%}
        \item Participated in 24/7 on-call rotations, triaging P1/P2 incidents and performing\textbf{ root cause analysis}, reducing mean time to resolution (MTTR) by \textbf{35\%} through improved runbooks and automated diagnostics
        \item Troubleshot \textbf{Jenkins} pipeline issues through root cause analysis, minimizing support ticket resolution time by \textbf{40\%} and ensuring \textbf{99.9\%} uptime for \textbf{CI/CD} workflows, leading to uninterrupted deployment pipelines
        \item Managed \textbf{300+} bi-weekly microservice deployments, using \textbf{Chef}, while also tracking them with \textbf{JIRA}, accelerating delivery of new UI and backend features in a fast-paced production environment
    \end{itemize}

\end{experienceSection}

% --------- Projects -----------
\begin{experienceSection}{Academic Projects}

    \projectItem[
        title={TalkDoc - AI-based Document Filler | Python, FastAPI, React, OpenAI GPT-4 | \href{https://github.com/Ajtambad/TalkDoc}{GitHub}},
        duration={Nov 2025},
    ]
    \begin{itemize}
        \vspace{-0.5em}
        \itemsep -6pt {}
        \item Built an AI-powered web application using \textbf{FastAPI} and \textbf{React} that automates document form completion through conversational interactions, leveraging\textbf{ OpenAI GPT-4} for natural language understanding and intelligent question generation
        \item Implemented dynamic conversation flows and document generation, reducing form completion time by \textbf{40\%} while achieving over \textbf{90\%} accuracy in mapping conversational responses to structured document fields
    \end{itemize}


    \projectItem[
        title={AWS-based Face Recognition App | Python, AWS | \href{https://github.com/Ajtambad/Cloud-Computing}{GitHub}},
        duration={May 2024},
    ]
    \begin{itemize}
        \vspace{-0.5em}
        \itemsep -6pt {}
        \item Developed and deployed a \textbf{Flask}-based image recognition app using \textbf{Gunicorn} on AWS EC2, enabling HTTP REST API based uploads and forwarding images to \textbf{S3} via \textbf{SQS} for asynchronous processing using AWS service API.
        \item Designed an auto-scaling app tier that scaled up to \textbf{20} EC2 instances based on \textbf{SQS} queue depth, ensuring efficient, real-time image processing under varying workloads
    \end{itemize}
    
\end{experienceSection}

% --------- Skills -----------
\begin{skillsSection}{Skills}
    \skillItem[
        category={Languages \& Scripting},
        skills={Python, Go, Bash, Java, SQL}
    ] \\
    \skillItem[
        category={Cloud \& Containers},
        skills={AWS, Docker, Kubernetes, Helm}
    ] \\
    \skillItem[
        category={Infrastructure as Code},
        skills={Terraform, Ansible, Chef, CloudFormation}
    ] \\
    \skillItem[
        category={CI/CD},
        skills={Jenkins, GitHub Actions, ArgoCD}
    ] \\
    \skillItem[
        category={Observability},
        skills={Splunk, Zabbix, Prometheus, Datadog, Grafana}
    ] \\
    \skillItem[
        category={Databases},
        skills={PostgreSQL, MySQL, MongoDB, Neo4j}
    ] \\
    \skillItem[
        category={Core Competency},
        skills={Distributed Systems, Microservices, Incident Management, System Design, Agile/Scrum}
    ]
\end{skillsSection}

\end{document}
